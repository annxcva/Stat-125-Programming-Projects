% Options for packages loaded elsewhere
\PassOptionsToPackage{unicode}{hyperref}
\PassOptionsToPackage{hyphens}{url}
%
\documentclass[
]{article}
\usepackage{amsmath,amssymb}
\usepackage{iftex}
\ifPDFTeX
  \usepackage[T1]{fontenc}
  \usepackage[utf8]{inputenc}
  \usepackage{textcomp} % provide euro and other symbols
\else % if luatex or xetex
  \usepackage{unicode-math} % this also loads fontspec
  \defaultfontfeatures{Scale=MatchLowercase}
  \defaultfontfeatures[\rmfamily]{Ligatures=TeX,Scale=1}
\fi
\usepackage{lmodern}
\ifPDFTeX\else
  % xetex/luatex font selection
\fi
% Use upquote if available, for straight quotes in verbatim environments
\IfFileExists{upquote.sty}{\usepackage{upquote}}{}
\IfFileExists{microtype.sty}{% use microtype if available
  \usepackage[]{microtype}
  \UseMicrotypeSet[protrusion]{basicmath} % disable protrusion for tt fonts
}{}
\makeatletter
\@ifundefined{KOMAClassName}{% if non-KOMA class
  \IfFileExists{parskip.sty}{%
    \usepackage{parskip}
  }{% else
    \setlength{\parindent}{0pt}
    \setlength{\parskip}{6pt plus 2pt minus 1pt}}
}{% if KOMA class
  \KOMAoptions{parskip=half}}
\makeatother
\usepackage{xcolor}
\usepackage[margin=1in]{geometry}
\usepackage{color}
\usepackage{fancyvrb}
\newcommand{\VerbBar}{|}
\newcommand{\VERB}{\Verb[commandchars=\\\{\}]}
\DefineVerbatimEnvironment{Highlighting}{Verbatim}{commandchars=\\\{\}}
% Add ',fontsize=\small' for more characters per line
\usepackage{framed}
\definecolor{shadecolor}{RGB}{248,248,248}
\newenvironment{Shaded}{\begin{snugshade}}{\end{snugshade}}
\newcommand{\AlertTok}[1]{\textcolor[rgb]{0.94,0.16,0.16}{#1}}
\newcommand{\AnnotationTok}[1]{\textcolor[rgb]{0.56,0.35,0.01}{\textbf{\textit{#1}}}}
\newcommand{\AttributeTok}[1]{\textcolor[rgb]{0.13,0.29,0.53}{#1}}
\newcommand{\BaseNTok}[1]{\textcolor[rgb]{0.00,0.00,0.81}{#1}}
\newcommand{\BuiltInTok}[1]{#1}
\newcommand{\CharTok}[1]{\textcolor[rgb]{0.31,0.60,0.02}{#1}}
\newcommand{\CommentTok}[1]{\textcolor[rgb]{0.56,0.35,0.01}{\textit{#1}}}
\newcommand{\CommentVarTok}[1]{\textcolor[rgb]{0.56,0.35,0.01}{\textbf{\textit{#1}}}}
\newcommand{\ConstantTok}[1]{\textcolor[rgb]{0.56,0.35,0.01}{#1}}
\newcommand{\ControlFlowTok}[1]{\textcolor[rgb]{0.13,0.29,0.53}{\textbf{#1}}}
\newcommand{\DataTypeTok}[1]{\textcolor[rgb]{0.13,0.29,0.53}{#1}}
\newcommand{\DecValTok}[1]{\textcolor[rgb]{0.00,0.00,0.81}{#1}}
\newcommand{\DocumentationTok}[1]{\textcolor[rgb]{0.56,0.35,0.01}{\textbf{\textit{#1}}}}
\newcommand{\ErrorTok}[1]{\textcolor[rgb]{0.64,0.00,0.00}{\textbf{#1}}}
\newcommand{\ExtensionTok}[1]{#1}
\newcommand{\FloatTok}[1]{\textcolor[rgb]{0.00,0.00,0.81}{#1}}
\newcommand{\FunctionTok}[1]{\textcolor[rgb]{0.13,0.29,0.53}{\textbf{#1}}}
\newcommand{\ImportTok}[1]{#1}
\newcommand{\InformationTok}[1]{\textcolor[rgb]{0.56,0.35,0.01}{\textbf{\textit{#1}}}}
\newcommand{\KeywordTok}[1]{\textcolor[rgb]{0.13,0.29,0.53}{\textbf{#1}}}
\newcommand{\NormalTok}[1]{#1}
\newcommand{\OperatorTok}[1]{\textcolor[rgb]{0.81,0.36,0.00}{\textbf{#1}}}
\newcommand{\OtherTok}[1]{\textcolor[rgb]{0.56,0.35,0.01}{#1}}
\newcommand{\PreprocessorTok}[1]{\textcolor[rgb]{0.56,0.35,0.01}{\textit{#1}}}
\newcommand{\RegionMarkerTok}[1]{#1}
\newcommand{\SpecialCharTok}[1]{\textcolor[rgb]{0.81,0.36,0.00}{\textbf{#1}}}
\newcommand{\SpecialStringTok}[1]{\textcolor[rgb]{0.31,0.60,0.02}{#1}}
\newcommand{\StringTok}[1]{\textcolor[rgb]{0.31,0.60,0.02}{#1}}
\newcommand{\VariableTok}[1]{\textcolor[rgb]{0.00,0.00,0.00}{#1}}
\newcommand{\VerbatimStringTok}[1]{\textcolor[rgb]{0.31,0.60,0.02}{#1}}
\newcommand{\WarningTok}[1]{\textcolor[rgb]{0.56,0.35,0.01}{\textbf{\textit{#1}}}}
\usepackage{graphicx}
\makeatletter
\def\maxwidth{\ifdim\Gin@nat@width>\linewidth\linewidth\else\Gin@nat@width\fi}
\def\maxheight{\ifdim\Gin@nat@height>\textheight\textheight\else\Gin@nat@height\fi}
\makeatother
% Scale images if necessary, so that they will not overflow the page
% margins by default, and it is still possible to overwrite the defaults
% using explicit options in \includegraphics[width, height, ...]{}
\setkeys{Gin}{width=\maxwidth,height=\maxheight,keepaspectratio}
% Set default figure placement to htbp
\makeatletter
\def\fps@figure{htbp}
\makeatother
\setlength{\emergencystretch}{3em} % prevent overfull lines
\providecommand{\tightlist}{%
  \setlength{\itemsep}{0pt}\setlength{\parskip}{0pt}}
\setcounter{secnumdepth}{-\maxdimen} % remove section numbering
\ifLuaTeX
  \usepackage{selnolig}  % disable illegal ligatures
\fi
\usepackage{bookmark}
\IfFileExists{xurl.sty}{\usepackage{xurl}}{} % add URL line breaks if available
\urlstyle{same}
\hypersetup{
  pdftitle={Car Prices},
  pdfauthor={Charlie's Angels},
  hidelinks,
  pdfcreator={LaTeX via pandoc}}

\title{Car Prices}
\author{Charlie's Angels}
\date{2024-05-26}

\begin{document}
\maketitle

\begin{Shaded}
\begin{Highlighting}[]
\NormalTok{knitr}\SpecialCharTok{::}\NormalTok{opts\_chunk}\SpecialCharTok{$}\FunctionTok{set}\NormalTok{(}\AttributeTok{warning =} \ConstantTok{FALSE}\NormalTok{, }\AttributeTok{message =} \ConstantTok{FALSE}\NormalTok{)}
\end{Highlighting}
\end{Shaded}

\begin{Shaded}
\begin{Highlighting}[]
\FunctionTok{library}\NormalTok{(readr)}
\FunctionTok{library}\NormalTok{(readxl)}
\FunctionTok{library}\NormalTok{(dplyr)}
\FunctionTok{library}\NormalTok{(ggplot2)}
\end{Highlighting}
\end{Shaded}

\begin{Shaded}
\begin{Highlighting}[]
\NormalTok{carprice }\OtherTok{\textless{}{-}} \FunctionTok{read\_xlsx}\NormalTok{(}\AttributeTok{path =} \StringTok{"CAR PRICE.xlsx"}\NormalTok{, }\AttributeTok{sheet =} \StringTok{"Data"}\NormalTok{)}
\FunctionTok{head}\NormalTok{(carprice)}
\end{Highlighting}
\end{Shaded}

\begin{verbatim}
## # A tibble: 6 x 26
##   car_ID symboling CarName     fueltype aspiration doornumber carbody drivewheel
##    <dbl>     <dbl> <chr>       <chr>    <chr>      <chr>      <chr>   <chr>     
## 1     75         1 buick rega~ gas      std        two        hardtop rwd       
## 2     17         0 bmw x5      gas      std        two        sedan   rwd       
## 3     74         0 buick cent~ gas      std        four       sedan   rwd       
## 4    129         3 porsche bo~ gas      std        two        conver~ rwd       
## 5     18         0 bmw x3      gas      std        four       sedan   rwd       
## 6     50         0 jaguar xk   gas      std        two        sedan   rwd       
## # i 18 more variables: enginelocation <chr>, wheelbase <dbl>, carlength <dbl>,
## #   carwidth <dbl>, carheight <dbl>, curbweight <dbl>, enginetype <chr>,
## #   cylindernumber <chr>, enginesize <dbl>, fuelsystem <chr>, boreratio <dbl>,
## #   stroke <dbl>, compressionratio <dbl>, horsepower <dbl>, peakrpm <dbl>,
## #   citympg <dbl>, highwaympg <dbl>, price <dbl>
\end{verbatim}

\subsection{A. Subset or ``split'' the carprice into 2
datasets:}\label{a.-subset-or-split-the-carprice-into-2-datasets}

\begin{itemize}
\tightlist
\item
  train: contains 150 randomly selected cars from the original dataset
\item
  test: contains the other 55 not selected in the train set. Use 125 as
  your seed number
\end{itemize}

\begin{Shaded}
\begin{Highlighting}[]
\FunctionTok{set.seed}\NormalTok{(}\DecValTok{125}\NormalTok{)}
\NormalTok{train\_samp }\OtherTok{\textless{}{-}} \FunctionTok{sample}\NormalTok{(}\FunctionTok{nrow}\NormalTok{(carprice), }\DecValTok{150}\NormalTok{)}

\NormalTok{train }\OtherTok{\textless{}{-}}\NormalTok{ carprice[train\_samp,]}
\NormalTok{test }\OtherTok{\textless{}{-}}\NormalTok{ carprice[}\SpecialCharTok{{-}}\NormalTok{train\_samp,]}

\FunctionTok{head}\NormalTok{(train)}
\end{Highlighting}
\end{Shaded}

\begin{verbatim}
## # A tibble: 6 x 26
##   car_ID symboling CarName     fueltype aspiration doornumber carbody drivewheel
##    <dbl>     <dbl> <chr>       <chr>    <chr>      <chr>      <chr>   <chr>     
## 1      8         1 audi 5000   gas      std        four       wagon   fwd       
## 2    152         1 toyota cor~ gas      std        two        hatchb~ fwd       
## 3      1         3 alfa-romer~ gas      std        two        conver~ rwd       
## 4    203        -1 volvo 244dl gas      std        four       sedan   rwd       
## 5     53         1 mazda rx2 ~ gas      std        two        hatchb~ fwd       
## 6    163         0 toyota mar~ gas      std        four       sedan   fwd       
## # i 18 more variables: enginelocation <chr>, wheelbase <dbl>, carlength <dbl>,
## #   carwidth <dbl>, carheight <dbl>, curbweight <dbl>, enginetype <chr>,
## #   cylindernumber <chr>, enginesize <dbl>, fuelsystem <chr>, boreratio <dbl>,
## #   stroke <dbl>, compressionratio <dbl>, horsepower <dbl>, peakrpm <dbl>,
## #   citympg <dbl>, highwaympg <dbl>, price <dbl>
\end{verbatim}

\begin{Shaded}
\begin{Highlighting}[]
\FunctionTok{head}\NormalTok{(test)}
\end{Highlighting}
\end{Shaded}

\begin{verbatim}
## # A tibble: 6 x 26
##   car_ID symboling CarName     fueltype aspiration doornumber carbody drivewheel
##    <dbl>     <dbl> <chr>       <chr>    <chr>      <chr>      <chr>   <chr>     
## 1     74         0 buick cent~ gas      std        four       sedan   rwd       
## 2     49         0 jaguar xf   gas      std        four       sedan   rwd       
## 3    130         1 porsche ca~ gas      std        two        hatchb~ rwd       
## 4    205        -1 volvo 264gl gas      turbo      four       sedan   rwd       
## 5    106         3 nissan kic~ gas      turbo      two        hatchb~ rwd       
## 6    202        -1 volvo 144ea gas      turbo      four       sedan   rwd       
## # i 18 more variables: enginelocation <chr>, wheelbase <dbl>, carlength <dbl>,
## #   carwidth <dbl>, carheight <dbl>, curbweight <dbl>, enginetype <chr>,
## #   cylindernumber <chr>, enginesize <dbl>, fuelsystem <chr>, boreratio <dbl>,
## #   stroke <dbl>, compressionratio <dbl>, horsepower <dbl>, peakrpm <dbl>,
## #   citympg <dbl>, highwaympg <dbl>, price <dbl>
\end{verbatim}

\subsection{B. Using the variables you selected in MP1, fit a multiple
linear regression model using the train
dataset.}\label{b.-using-the-variables-you-selected-in-mp1-fit-a-multiple-linear-regression-model-using-the-train-dataset.}

Store the lm class object to an object named model\_1. Show results
using summary(model\_1).

\begin{Shaded}
\begin{Highlighting}[]
\NormalTok{model\_1 }\OtherTok{\textless{}{-}} \FunctionTok{lm}\NormalTok{(price }\SpecialCharTok{\textasciitilde{}}\NormalTok{ enginesize }\SpecialCharTok{+}\NormalTok{ peakrpm }\SpecialCharTok{+}\NormalTok{ boreratio, }\AttributeTok{data =}\NormalTok{ train)}
\FunctionTok{summary}\NormalTok{(model\_1)}
\end{Highlighting}
\end{Shaded}

\begin{verbatim}
## 
## Call:
## lm(formula = price ~ enginesize + peakrpm + boreratio, data = train)
## 
## Residuals:
##     Min      1Q  Median      3Q     Max 
## -694353  -92401  -27063  100377  708904 
## 
## Coefficients:
##               Estimate Std. Error t value Pr(>|t|)    
## (Intercept) -1.300e+06  3.495e+05  -3.719 0.000285 ***
## enginesize   9.847e+03  5.312e+02  18.537  < 2e-16 ***
## peakrpm      1.002e+02  3.746e+01   2.674 0.008354 ** 
## boreratio    8.479e+04  8.483e+04   1.000 0.319173    
## ---
## Signif. codes:  0 '***' 0.001 '**' 0.01 '*' 0.05 '.' 0.1 ' ' 1
## 
## Residual standard error: 212100 on 146 degrees of freedom
## Multiple R-squared:  0.7943, Adjusted R-squared:  0.7901 
## F-statistic: 187.9 on 3 and 146 DF,  p-value: < 2.2e-16
\end{verbatim}

\begin{Shaded}
\begin{Highlighting}[]
\FunctionTok{ggplot}\NormalTok{(train, }\FunctionTok{aes}\NormalTok{(}\AttributeTok{y =}\NormalTok{ price, }\AttributeTok{x =}\NormalTok{ enginesize, }\AttributeTok{color =}\NormalTok{ peakrpm, }\AttributeTok{fill =}\NormalTok{ boreratio)) }\SpecialCharTok{+}
  \FunctionTok{geom\_point}\NormalTok{() }\SpecialCharTok{+}
  \FunctionTok{geom\_smooth}\NormalTok{(}\AttributeTok{method =}\NormalTok{ lm, }\AttributeTok{se =}\NormalTok{ F, }\AttributeTok{color =} \StringTok{\textquotesingle{}orange\textquotesingle{}}\NormalTok{) }\SpecialCharTok{+}
  \FunctionTok{scale\_color\_viridis\_b}\NormalTok{()}\SpecialCharTok{+}
  \FunctionTok{theme\_bw}\NormalTok{()}
\end{Highlighting}
\end{Shaded}

\includegraphics{3-CarPrices-\%5BCharlie-s-Angels\%5D_files/figure-latex/unnamed-chunk-5-1.pdf}

\subsection{C. Using model\_1, predict the prices in the test
dataset.}\label{c.-using-model_1-predict-the-prices-in-the-test-dataset.}

Store the vector of predicted values in an object named fit\_1.

\begin{Shaded}
\begin{Highlighting}[]
\NormalTok{fit\_1 }\OtherTok{\textless{}{-}} \FunctionTok{predict}\NormalTok{(model\_1, test)}

\NormalTok{data\_test }\OtherTok{\textless{}{-}} \FunctionTok{data.frame}\NormalTok{(}\StringTok{"Actual price"} \OtherTok{=}\NormalTok{ test}\SpecialCharTok{$}\NormalTok{price, }\StringTok{"Predicted price"} \OtherTok{=}\NormalTok{ fit\_1, }\StringTok{"Residuals"} \OtherTok{=}\NormalTok{ test}\SpecialCharTok{$}\NormalTok{price }\SpecialCharTok{{-}}\NormalTok{ fit\_1, }\StringTok{"Residuals Squared"} \OtherTok{=}\NormalTok{ (test}\SpecialCharTok{$}\NormalTok{price }\SpecialCharTok{{-}}\NormalTok{ fit\_1)}\SpecialCharTok{\^{}}\DecValTok{2}\NormalTok{ )}
\FunctionTok{head}\NormalTok{(data\_test)}
\end{Highlighting}
\end{Shaded}

\begin{verbatim}
##   Actual.price Predicted.price  Residuals Residuals.Squared
## 1      2295000       2506069.8 -211069.84       44550477966
## 2      1992000       2024343.5  -32343.53        1046104250
## 3      1760000       1609203.3  150796.71       22739647732
## 4      1268000        950062.4  317937.57      101084298178
## 5      1104000       1294235.6 -190235.63       36189593586
## 6      1067000        940046.1  126953.87       16117284322
\end{verbatim}

\subsection{D. In statistical modelling, the performance is evaluated
using some accuracy metrics, such as the Root Mean Square Error
(RMSE)}\label{d.-in-statistical-modelling-the-performance-is-evaluated-using-some-accuracy-metrics-such-as-the-root-mean-square-error-rmse}

\begin{Shaded}
\begin{Highlighting}[]
\NormalTok{rmse\_manual }\OtherTok{\textless{}{-}} \FunctionTok{sqrt}\NormalTok{(}\FunctionTok{sum}\NormalTok{(data\_test}\SpecialCharTok{$}\NormalTok{Residuals.Squared)}\SpecialCharTok{/}\FunctionTok{nrow}\NormalTok{(data\_test))}
\NormalTok{rmse\_manual}
\end{Highlighting}
\end{Shaded}

\begin{verbatim}
## [1] 207745.8
\end{verbatim}

\begin{Shaded}
\begin{Highlighting}[]
\NormalTok{rmse\_fun }\OtherTok{\textless{}{-}}\NormalTok{ Metrics}\SpecialCharTok{::}\FunctionTok{rmse}\NormalTok{(}\AttributeTok{actual =}\NormalTok{ test}\SpecialCharTok{$}\NormalTok{price, }\AttributeTok{predicted =}\NormalTok{ fit\_1)}
\NormalTok{rmse\_fun}
\end{Highlighting}
\end{Shaded}

\begin{verbatim}
## [1] 207745.8
\end{verbatim}

\textbf{Root Mean Square Error = 203229.5}

\end{document}
