% Options for packages loaded elsewhere
\PassOptionsToPackage{unicode}{hyperref}
\PassOptionsToPackage{hyphens}{url}
%
\documentclass[
]{article}
\usepackage{amsmath,amssymb}
\usepackage{iftex}
\ifPDFTeX
  \usepackage[T1]{fontenc}
  \usepackage[utf8]{inputenc}
  \usepackage{textcomp} % provide euro and other symbols
\else % if luatex or xetex
  \usepackage{unicode-math} % this also loads fontspec
  \defaultfontfeatures{Scale=MatchLowercase}
  \defaultfontfeatures[\rmfamily]{Ligatures=TeX,Scale=1}
\fi
\usepackage{lmodern}
\ifPDFTeX\else
  % xetex/luatex font selection
\fi
% Use upquote if available, for straight quotes in verbatim environments
\IfFileExists{upquote.sty}{\usepackage{upquote}}{}
\IfFileExists{microtype.sty}{% use microtype if available
  \usepackage[]{microtype}
  \UseMicrotypeSet[protrusion]{basicmath} % disable protrusion for tt fonts
}{}
\makeatletter
\@ifundefined{KOMAClassName}{% if non-KOMA class
  \IfFileExists{parskip.sty}{%
    \usepackage{parskip}
  }{% else
    \setlength{\parindent}{0pt}
    \setlength{\parskip}{6pt plus 2pt minus 1pt}}
}{% if KOMA class
  \KOMAoptions{parskip=half}}
\makeatother
\usepackage{xcolor}
\usepackage[margin=1in]{geometry}
\usepackage{color}
\usepackage{fancyvrb}
\newcommand{\VerbBar}{|}
\newcommand{\VERB}{\Verb[commandchars=\\\{\}]}
\DefineVerbatimEnvironment{Highlighting}{Verbatim}{commandchars=\\\{\}}
% Add ',fontsize=\small' for more characters per line
\usepackage{framed}
\definecolor{shadecolor}{RGB}{248,248,248}
\newenvironment{Shaded}{\begin{snugshade}}{\end{snugshade}}
\newcommand{\AlertTok}[1]{\textcolor[rgb]{0.94,0.16,0.16}{#1}}
\newcommand{\AnnotationTok}[1]{\textcolor[rgb]{0.56,0.35,0.01}{\textbf{\textit{#1}}}}
\newcommand{\AttributeTok}[1]{\textcolor[rgb]{0.13,0.29,0.53}{#1}}
\newcommand{\BaseNTok}[1]{\textcolor[rgb]{0.00,0.00,0.81}{#1}}
\newcommand{\BuiltInTok}[1]{#1}
\newcommand{\CharTok}[1]{\textcolor[rgb]{0.31,0.60,0.02}{#1}}
\newcommand{\CommentTok}[1]{\textcolor[rgb]{0.56,0.35,0.01}{\textit{#1}}}
\newcommand{\CommentVarTok}[1]{\textcolor[rgb]{0.56,0.35,0.01}{\textbf{\textit{#1}}}}
\newcommand{\ConstantTok}[1]{\textcolor[rgb]{0.56,0.35,0.01}{#1}}
\newcommand{\ControlFlowTok}[1]{\textcolor[rgb]{0.13,0.29,0.53}{\textbf{#1}}}
\newcommand{\DataTypeTok}[1]{\textcolor[rgb]{0.13,0.29,0.53}{#1}}
\newcommand{\DecValTok}[1]{\textcolor[rgb]{0.00,0.00,0.81}{#1}}
\newcommand{\DocumentationTok}[1]{\textcolor[rgb]{0.56,0.35,0.01}{\textbf{\textit{#1}}}}
\newcommand{\ErrorTok}[1]{\textcolor[rgb]{0.64,0.00,0.00}{\textbf{#1}}}
\newcommand{\ExtensionTok}[1]{#1}
\newcommand{\FloatTok}[1]{\textcolor[rgb]{0.00,0.00,0.81}{#1}}
\newcommand{\FunctionTok}[1]{\textcolor[rgb]{0.13,0.29,0.53}{\textbf{#1}}}
\newcommand{\ImportTok}[1]{#1}
\newcommand{\InformationTok}[1]{\textcolor[rgb]{0.56,0.35,0.01}{\textbf{\textit{#1}}}}
\newcommand{\KeywordTok}[1]{\textcolor[rgb]{0.13,0.29,0.53}{\textbf{#1}}}
\newcommand{\NormalTok}[1]{#1}
\newcommand{\OperatorTok}[1]{\textcolor[rgb]{0.81,0.36,0.00}{\textbf{#1}}}
\newcommand{\OtherTok}[1]{\textcolor[rgb]{0.56,0.35,0.01}{#1}}
\newcommand{\PreprocessorTok}[1]{\textcolor[rgb]{0.56,0.35,0.01}{\textit{#1}}}
\newcommand{\RegionMarkerTok}[1]{#1}
\newcommand{\SpecialCharTok}[1]{\textcolor[rgb]{0.81,0.36,0.00}{\textbf{#1}}}
\newcommand{\SpecialStringTok}[1]{\textcolor[rgb]{0.31,0.60,0.02}{#1}}
\newcommand{\StringTok}[1]{\textcolor[rgb]{0.31,0.60,0.02}{#1}}
\newcommand{\VariableTok}[1]{\textcolor[rgb]{0.00,0.00,0.00}{#1}}
\newcommand{\VerbatimStringTok}[1]{\textcolor[rgb]{0.31,0.60,0.02}{#1}}
\newcommand{\WarningTok}[1]{\textcolor[rgb]{0.56,0.35,0.01}{\textbf{\textit{#1}}}}
\usepackage{graphicx}
\makeatletter
\def\maxwidth{\ifdim\Gin@nat@width>\linewidth\linewidth\else\Gin@nat@width\fi}
\def\maxheight{\ifdim\Gin@nat@height>\textheight\textheight\else\Gin@nat@height\fi}
\makeatother
% Scale images if necessary, so that they will not overflow the page
% margins by default, and it is still possible to overwrite the defaults
% using explicit options in \includegraphics[width, height, ...]{}
\setkeys{Gin}{width=\maxwidth,height=\maxheight,keepaspectratio}
% Set default figure placement to htbp
\makeatletter
\def\fps@figure{htbp}
\makeatother
\setlength{\emergencystretch}{3em} % prevent overfull lines
\providecommand{\tightlist}{%
  \setlength{\itemsep}{0pt}\setlength{\parskip}{0pt}}
\setcounter{secnumdepth}{-\maxdimen} % remove section numbering
\ifLuaTeX
  \usepackage{selnolig}  % disable illegal ligatures
\fi
\usepackage{bookmark}
\IfFileExists{xurl.sty}{\usepackage{xurl}}{} % add URL line breaks if available
\urlstyle{same}
\hypersetup{
  pdftitle={Weight Loss},
  pdfauthor={Charlie's Angels},
  hidelinks,
  pdfcreator={LaTeX via pandoc}}

\title{Weight Loss}
\author{Charlie's Angels}
\date{2024-05-29}

\begin{document}
\maketitle

\begin{Shaded}
\begin{Highlighting}[]
\NormalTok{knitr}\SpecialCharTok{::}\NormalTok{opts\_chunk}\SpecialCharTok{$}\FunctionTok{set}\NormalTok{(}\AttributeTok{warning =} \ConstantTok{FALSE}\NormalTok{, }\AttributeTok{message =} \ConstantTok{FALSE}\NormalTok{)}
\end{Highlighting}
\end{Shaded}

\section{A. Gym as a factor in Weight
Loss}\label{a.-gym-as-a-factor-in-weight-loss}

Question \#1: On the average, which Gym has individuals with higher
weight loss?

\begin{Shaded}
\begin{Highlighting}[]
\FunctionTok{library}\NormalTok{(readr)}
\FunctionTok{library}\NormalTok{(dplyr)}
\FunctionTok{library}\NormalTok{(rstatix)}
\FunctionTok{library}\NormalTok{(tinytex)}
\FunctionTok{library}\NormalTok{(ggplot2)}

\NormalTok{typediet }\OtherTok{\textless{}{-}} \FunctionTok{c}\NormalTok{(}\StringTok{"A"}\NormalTok{, }\StringTok{"B"}\NormalTok{, }\StringTok{"C"}\NormalTok{)}
\NormalTok{typegym }\OtherTok{\textless{}{-}} \FunctionTok{c}\NormalTok{(}\StringTok{"Pewter"}\NormalTok{, }\StringTok{"Cerulean"}\NormalTok{)}


\NormalTok{WEIGHTLOSS }\OtherTok{\textless{}{-}} \FunctionTok{read\_csv}\NormalTok{(}\StringTok{"WeightLoss.csv"}\NormalTok{, }
                       \AttributeTok{col\_types =} \FunctionTok{cols}\NormalTok{(}\AttributeTok{Diet =} \FunctionTok{col\_factor}\NormalTok{(}\AttributeTok{levels =}\NormalTok{ typediet), }
                                        \AttributeTok{Gym =} \FunctionTok{col\_factor}\NormalTok{(}\AttributeTok{levels =}\NormalTok{ typegym)}
\NormalTok{                                       )}
\NormalTok{                      )}
\FunctionTok{head}\NormalTok{(WEIGHTLOSS)}
\end{Highlighting}
\end{Shaded}

\begin{verbatim}
## # A tibble: 6 x 5
##   MemberID   Age Diet  Gym      WeightLoss
##      <dbl> <dbl> <fct> <fct>         <dbl>
## 1        1    35 A     Cerulean       5.74
## 2        2    29 C     Cerulean       7.36
## 3        3    27 B     Pewter         7.17
## 4        4    23 C     Pewter        11.9 
## 5        5    26 B     Pewter         8.78
## 6        6    32 B     Cerulean       5.52
\end{verbatim}

\begin{Shaded}
\begin{Highlighting}[]
\NormalTok{mean\_gym }\OtherTok{\textless{}{-}}\NormalTok{ WEIGHTLOSS }\SpecialCharTok{\%\textgreater{}\%}
  \FunctionTok{group\_by}\NormalTok{(Gym) }\SpecialCharTok{\%\textgreater{}\%}
  \FunctionTok{summarize}\NormalTok{(}\AttributeTok{mean\_value =} \FunctionTok{mean}\NormalTok{(WeightLoss, }\AttributeTok{na.rm =} \ConstantTok{TRUE}\NormalTok{))}
\NormalTok{mean\_gym}
\end{Highlighting}
\end{Shaded}

\begin{verbatim}
## # A tibble: 2 x 2
##   Gym      mean_value
##   <fct>         <dbl>
## 1 Pewter         8.01
## 2 Cerulean       6.04
\end{verbatim}

\begin{Shaded}
\begin{Highlighting}[]
\FunctionTok{ggplot}\NormalTok{(WEIGHTLOSS, }\FunctionTok{aes}\NormalTok{(}\AttributeTok{x =}\NormalTok{ Gym, }\AttributeTok{y =}\NormalTok{ WeightLoss, }\AttributeTok{fill =}\NormalTok{ Gym)) }\SpecialCharTok{+}
  \FunctionTok{geom\_boxplot}\NormalTok{() }\SpecialCharTok{+}
  \FunctionTok{labs}\NormalTok{(}\AttributeTok{title =} \StringTok{"Weight Loss by Diet"}\NormalTok{, }\AttributeTok{x =} \StringTok{"Gym"}\NormalTok{, }\AttributeTok{y =} \StringTok{"Weight Lost (in kg)"}\NormalTok{)}
\end{Highlighting}
\end{Shaded}

\includegraphics{2-WeightLoss-\%5BCharlie-s-Angels\%5D_files/figure-latex/unnamed-chunk-4-1.pdf}

\textbf{Answer: On average, Pewter gym has individuals with higher
weight loss}

Question \#2: At 0.05 level of significance, is there a difference in
the average weight loss between the members of the two Gyms? Test for
assumptions before performing the T test for means.

\begin{Shaded}
\begin{Highlighting}[]
\FunctionTok{library}\NormalTok{(rstatix)}
\FunctionTok{library}\NormalTok{(tidyverse)}

\NormalTok{WL2 }\OtherTok{\textless{}{-}} \FunctionTok{subset}\NormalTok{(WEIGHTLOSS, Gym }\SpecialCharTok{\%in\%} \FunctionTok{c}\NormalTok{(}\StringTok{"Pewter"}\NormalTok{, }\StringTok{"Cerulean"}\NormalTok{))}\SpecialCharTok{|\textgreater{}}
  \FunctionTok{select}\NormalTok{(WeightLoss, Gym)}\SpecialCharTok{|\textgreater{}}
  \FunctionTok{mutate}\NormalTok{(}\AttributeTok{Gym =} \FunctionTok{as.factor}\NormalTok{(Gym))}
\NormalTok{WL2}
\end{Highlighting}
\end{Shaded}

\begin{verbatim}
## # A tibble: 60 x 2
##    WeightLoss Gym     
##         <dbl> <fct>   
##  1       5.74 Cerulean
##  2       7.36 Cerulean
##  3       7.17 Pewter  
##  4      11.9  Pewter  
##  5       8.78 Pewter  
##  6       5.52 Cerulean
##  7      10.2  Pewter  
##  8       7.77 Cerulean
##  9       7.79 Pewter  
## 10       5.49 Cerulean
## # i 50 more rows
\end{verbatim}

\begin{Shaded}
\begin{Highlighting}[]
\FunctionTok{shapiro.test}\NormalTok{(WL2}\SpecialCharTok{$}\NormalTok{WeightLoss)}
\end{Highlighting}
\end{Shaded}

\begin{verbatim}
## 
##  Shapiro-Wilk normality test
## 
## data:  WL2$WeightLoss
## W = 0.97623, p-value = 0.2902
\end{verbatim}

\begin{Shaded}
\begin{Highlighting}[]
\NormalTok{summary\_gym }\OtherTok{\textless{}{-}}\NormalTok{ WL2}\SpecialCharTok{\%\textgreater{}\%}
  \FunctionTok{group\_by}\NormalTok{(Gym)}\SpecialCharTok{\%\textgreater{}\%}
  \FunctionTok{summarize}\NormalTok{(}\FunctionTok{shapiro\_test}\NormalTok{(WeightLoss))}
\NormalTok{summary\_gym}
\end{Highlighting}
\end{Shaded}

\begin{verbatim}
## # A tibble: 2 x 4
##   Gym      variable   statistic p.value
##   <fct>    <chr>          <dbl>   <dbl>
## 1 Pewter   WeightLoss     0.967   0.468
## 2 Cerulean WeightLoss     0.958   0.282
\end{verbatim}

\emph{The weight loss data from the 2 gyms follows a normal distribution
since the p-values of the 2 gyms are greater than 0.05}

\begin{Shaded}
\begin{Highlighting}[]
\FunctionTok{t.test}\NormalTok{(WeightLoss }\SpecialCharTok{\textasciitilde{}}\NormalTok{ Gym, WL2,}
       \AttributeTok{var.equal =}\NormalTok{ F,}
       \AttributeTok{alternative =} \StringTok{"two.sided"}\NormalTok{)}
\end{Highlighting}
\end{Shaded}

\begin{verbatim}
## 
##  Welch Two Sample t-test
## 
## data:  WeightLoss by Gym
## t = 4.333, df = 43.356, p-value = 8.58e-05
## alternative hypothesis: true difference in means between group Pewter and group Cerulean is not equal to 0
## 95 percent confidence interval:
##  1.050836 2.879830
## sample estimates:
##   mean in group Pewter mean in group Cerulean 
##               8.009333               6.044000
\end{verbatim}

\textbf{Answer: At 0.05 level of significance, we have sufficient
evidence to conclude that the average weight loss between the members of
the 2 gyms are not equal}

\section{B. Diet as a factor in Weight
Loss}\label{b.-diet-as-a-factor-in-weight-loss}

Question \#3: Obtain mean of WeightLoss per Diet. Which types of diet
have greater mean weight loss than the overall mean weight loss? Which
types of diet have less mean weight loss than the overall mean weight
loss?

\begin{Shaded}
\begin{Highlighting}[]
\NormalTok{mean\_diet }\OtherTok{\textless{}{-}}\NormalTok{ WEIGHTLOSS }\SpecialCharTok{\%\textgreater{}\%}
  \FunctionTok{group\_by}\NormalTok{(Diet) }\SpecialCharTok{\%\textgreater{}\%}
  \FunctionTok{summarize}\NormalTok{(}\AttributeTok{mean\_diet =} \FunctionTok{mean}\NormalTok{(WeightLoss))}
\NormalTok{mean\_diet}
\end{Highlighting}
\end{Shaded}

\begin{verbatim}
## # A tibble: 3 x 2
##   Diet  mean_diet
##   <fct>     <dbl>
## 1 A          5.82
## 2 B          7.05
## 3 C          8.21
\end{verbatim}

\begin{Shaded}
\begin{Highlighting}[]
\NormalTok{OVmean }\OtherTok{\textless{}{-}}\NormalTok{ WEIGHTLOSS }\SpecialCharTok{\%\textgreater{}\%} 
          \FunctionTok{summarize}\NormalTok{(}\AttributeTok{OVmean =} \FunctionTok{mean}\NormalTok{(WeightLoss))}
\NormalTok{OVmean}
\end{Highlighting}
\end{Shaded}

\begin{verbatim}
## # A tibble: 1 x 1
##   OVmean
##    <dbl>
## 1   7.03
\end{verbatim}

\textbf{Answer: Among the three diets, Diet B \& C has greater mean than
the overall mean weight loss. On the other hand, Diet A is the only diet
that has less mean weight loss than the overall mean weight loss.}

Question \#4: Perform a one-way ANOVA of WeightLoss with the type of
diet as the grouping variable. Check first if assumptions are met.
Interpret the result.

\begin{Shaded}
\begin{Highlighting}[]
\CommentTok{\# Test for Normality}
\NormalTok{WEIGHTLOSS }\SpecialCharTok{\%\textgreater{}\%} \FunctionTok{group\_by}\NormalTok{(Diet) }\SpecialCharTok{\%\textgreater{}\%}
  \FunctionTok{summarize}\NormalTok{(}\FunctionTok{shapiro\_test}\NormalTok{(WeightLoss))}
\end{Highlighting}
\end{Shaded}

\begin{verbatim}
## # A tibble: 3 x 4
##   Diet  variable   statistic p.value
##   <fct> <chr>          <dbl>   <dbl>
## 1 A     WeightLoss     0.971 0.768  
## 2 B     WeightLoss     0.971 0.781  
## 3 C     WeightLoss     0.854 0.00626
\end{verbatim}

\emph{The third group has a p-value \textless{} 0.05, violating the
assumption of normality. We proceed with testing for homoscedasticity.
We proceed to Levene's Test for Homoscedasticity}

\begin{Shaded}
\begin{Highlighting}[]
\CommentTok{\#Test for Homoscedasticity}
\FunctionTok{levene\_test}\NormalTok{( WEIGHTLOSS, WeightLoss }\SpecialCharTok{\textasciitilde{}}\NormalTok{ Diet)}
\end{Highlighting}
\end{Shaded}

\begin{verbatim}
## # A tibble: 1 x 4
##     df1   df2 statistic     p
##   <int> <int>     <dbl> <dbl>
## 1     2    57    0.0261 0.974
\end{verbatim}

\emph{Since the p-value is \textgreater{} 0.05, we do not reject the
null hypothesis. At 0.05 level of significance, we have sufficient
evidence to conclude that the variances are equal.}

We proceed with performing one-way ANOVA.

\begin{Shaded}
\begin{Highlighting}[]
\NormalTok{anova\_diet }\OtherTok{\textless{}{-}} \FunctionTok{aov}\NormalTok{(WeightLoss }\SpecialCharTok{\textasciitilde{}}\NormalTok{ Diet, WEIGHTLOSS)}
\FunctionTok{summary}\NormalTok{(anova\_diet)}
\end{Highlighting}
\end{Shaded}

\begin{verbatim}
##             Df Sum Sq Mean Sq F value   Pr(>F)    
## Diet         2  57.45  28.727   9.124 0.000365 ***
## Residuals   57 179.47   3.149                     
## ---
## Signif. codes:  0 '***' 0.001 '**' 0.01 '*' 0.05 '.' 0.1 ' ' 1
\end{verbatim}

\textbf{Answer: Since the p-value is \textless{} 0.05, we reject the
null hypothesis. At 0.05 level of significance, we have sufficient
evidence to conclude that at least one of the means of the 3 diets is
different from the rest.}

\end{document}
